\documentclass {article}
\usepackage{amsmath, amssymb, graphicx, geometry,float}

% GEOMETRY
\geometry{top=1.25in, bottom=1.25in, left=1.25in, right=1.25in}

\usepackage{floatrow}

% Background
\usepackage{background}
\backgroundsetup{
 scale=1,
 color=black,
 opacity=10,
 angle=0,
 contents={
  \includegraphics[width=\paperwidth,height=\paperheight]{brand.pdf}
 }
}
% Line Height
\renewcommand{\baselinestretch}{1}
% Par Skip and parindent
\setlength{\parindent}{0pt}
\setlength{\parskip}{0.5em}

\title{Population Dynamics Modeling: Dengue Infection in Bangladesh}
\author{\textbf{Name:} \textit{Abid Jeem}}
\date{\today}

\begin{document}

\maketitle

\section*{Model Selection}
\textbf{Chosen Model: SIR Model for Dengue Infection in Bangladesh}

\subsection*{Assumptions and Parameters}
\begin{itemize}
    \item The population is divided into Susceptible (S), Infected (I), and Recovered (R).
    \item Dengue spreads through mosquito bites, with transmission dependent on contact rates.
    \item Individuals recover from infection and gain temporary immunity.
    \item Birth and death rates of the human population are negligible for the short-term model.
    \item The mosquito population is assumed constant.
\end{itemize}

\subsection*{Mathematical Formulation}
The SIR model is described by the following differential equations:
\begin{equation}
\frac{dS}{dt} = -\beta \frac{S I}{N}
\end{equation}
\begin{equation}
\frac{dI}{dt} = \beta \frac{S I}{N} - \gamma I
\end{equation}
\begin{equation}
\frac{dR}{dt} = \gamma I
\end{equation}
where:
\begin{itemize}
    \item $S$, $I$, and $R$ represent the susceptible, infected, and recovered populations.
    \item $\beta$ is the transmission rate (probability of infection per contact).
    \item $\gamma$ is the recovery rate.
    \item $N$ is the total population size ($S + I + R$ is assumed constant).
\end{itemize}
The basic reproduction number, $R_0 = \frac{\beta}{\gamma}$, determines whether an outbreak spreads ($R_0 > 1$) or dies out ($R_0 < 1$).



\section*{Python Implementation}

\subsection*{Model Setup}
The SIR model was simulated with the following parameters:
\begin{itemize}
    \item Transmission rate: $\beta = 0.5$
    \item Recovery rate: $\gamma = \frac{1}{14}$
    \item Total population: $N = 1,000,000$
    \item Initial conditions: $S_0 = 999,000$, $I_0 = 1,000$, $R_0 = 0$
\end{itemize}
The simulation ran for 365 days to analyze outbreak progression.

\subsection*{Enhanced Model Features}
To refine the analysis, the model incorporated:
\begin{itemize}
    \item \textbf{Parameter Variability:} Simulating different $\beta$ values ($0.3, 0.5, 0.7$) to study outbreak severity.
    \item \textbf{Peak Infection Analysis:} Identifying peak infection time and magnitude.
    \item \textbf{Improved Visualization:} Labeling axes, adding legends, and annotating key points.
\end{itemize}

\begin{figure}[H]
    \centering
    \includegraphics[width=1\textwidth]{sir_model_plot.png}
    \caption{SIR Model Simulation for Dengue Infection in Bangladesh. The plot shows the evolution of infections for different transmission rates ($\beta$).}
    \label{fig:sir_model}
\end{figure}

\section*{Analysis and Discussion}

\subsection*{Key Observations}
The SIR model highlights the rise and decline of infections, with peak severity influenced by transmission rate ($\beta$). A higher $\beta$ results in an earlier, more intense outbreak, whereas a lower $\gamma$ prolongs the infectious period.

\subsection*{Impact of Parameter Variation}
Varying $\beta$ illustrates different outbreak dynamics:
\begin{itemize}
    \item \textbf{Lower $\beta$ (0.3):} Slower spread, prolonged outbreak, lower peak.
    \item \textbf{Baseline $\beta$ (0.5):} Moderate outbreak with a significant peak.
    \item \textbf{Higher $\beta$ (0.7):} Rapid spread, sharp peak, faster resolution.
\end{itemize}

\subsection*{Equilibrium and Stability}
The system trends toward a \textbf{disease-free equilibrium} as $I \to 0$, assuming no reinfection. No endemic equilibrium exists in this model.

\subsection*{Limitations}
While useful, the model has simplifying assumptions:
\begin{itemize}
    \item \textbf{Constant Population:} Ignores births, deaths, and migration.
    \item \textbf{No Seasonality:} Dengue outbreaks depend on mosquito activity, which varies seasonally.
    \item \textbf{Homogeneous Mixing:} Assumes equal interaction probability, ignoring geographic and social structures.
    \item \textbf{No Vector Dynamics:} Mosquito population dynamics are not modeled.
\end{itemize}

\end{document}

