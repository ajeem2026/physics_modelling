\documentclass {article}
\usepackage{amsmath, amssymb, graphicx, geometry}

% GEOMETERY
% sets paper size, margins and a parameter 
\geometry{top=1.5in, bottom=1.5in, left=1.5in, right=1.5in}

\usepackage{floatrow}




%Background
\usepackage{background}
\backgroundsetup{
 scale=1,
 color=black,
 opacity=10,
 angle=0,
 contents={
  \includegraphics[width=\paperwidth,height=\paperheight]{brand.pdf}
 }
}
%Line Height 
\renewcommand{\baselinestretch}{1.5} %Line spacing

%Par Skip and parindent
\setlength{\parindent}{0pt}
\setlength{\parskip} {1.2em}
\usetikzlibrary{arrows}
\title{Computational Worksheet: Exploring Oscillations}
\author{\textbf{Name:} \textit{Abid Jeem}}
\date{\today}

\begin{document}

\maketitle

\section*{Worksheet Overview}
\textbf{Goal:} Develop intuition and computational skills by modeling simple, damped, and driven damped oscillators using Python.

\textbf{Topics:}
\begin{itemize}
    \item Simple Harmonic Oscillator (SHO)
    \item Damped Oscillator
    \item Driven Damped Oscillator
\end{itemize}

\textbf{Please feel free to change the Python codes provided on Canvas  and the Worksheet Latex template to accommodate your analysis. Include all the relevant  plots in this worksheet. }\\

\textbf{To Submit to Canvas:} Completed worksheet with answers and plots.

\section*{Part 1: Simple Harmonic Oscillator (SHO)}
\subsection*{Theory}
The motion of a simple harmonic oscillator is governed by the equation:
\[ m \frac{d^2x}{dt^2} + kx = 0 \]

\subsection*{Questions}
\begin{enumerate}
    \item Write the equation in terms of two first-order differential equations. Show your work below:
    \[
\text{Equations:}
\]

\begin{equation}
v = \frac{dx}{dt}
\end{equation}

\begin{equation}
\frac{dv}{dt} = \frac{d^2 x}{dt^2}
\end{equation}

\[
\text{Hence, the second-order ODE becomes a system of two coupled first-order ODEs:}
\]

\begin{equation*}
\begin{aligned}
\text{Our coupled ODEs}
\begin{cases}
\dfrac{dx}{dt} = v,\\[6pt]
\dfrac{dv}{dt} = -\dfrac{k}{m}\, x
\end{cases}
\end{aligned}
\end{equation*}

    \item What is the analytical expression for the period of oscillation in terms of $m$ and $k$? Write it below:
    \vspace{1cm}
    \[ T = 2\pi \sqrt{\frac{m}{k}}
 \]
 \vspace{0.5cm}
    \item How does changing the mass $m$ or the spring constant $k$ affect the period? Explain briefly:

    \begin{itemize}
    
    \item Increasing \( m \) makes the period longer \((T \propto \sqrt{m})\)

    \item Increasing \( k \) makes the period shorter \((T \propto \frac{1}{\sqrt{k}})\)
    \end{itemize}
\end{enumerate}

\subsection*{Tasks}
\begin{enumerate}
    \item Simulate SHO for different values of $k$ and $m$ using the code provided.
    \item Plot position $x(t)$ and velocity $v(t)$ versus time.
    \begin{figure}[H]
    \centering
    \includegraphics[width=1.0\textwidth]{Figure_1.png}
    \caption{Position $x(t)$ and velocity $v(t)$ of the simple harmonic oscillator for varying values of $k$=10.0 and $m$=1.0.}
    \label{fig:sho_simulation}
    \end{figure}

    \begin{figure}[H]
        \centering
        \includegraphics[width=1.0\textwidth]{Figure_2.png}
        \caption{Position $x(t)$ and velocity $v(t)$ of the simple harmonic oscillator for varying values of $k$=20.0 and $m$=5.0.}
        \label{fig:sho_simulation}
    \end{figure}

    \begin{figure}[H]
        \centering
        \includegraphics[width=1.0\textwidth]{Figure_3.png}
        \caption{Position $x(t)$ and velocity $v(t)$ of the simple harmonic oscillator for varying values of $k$=30.0 and $m$=10.0.}
        \label{fig:sho_simulation}
    \end{figure}
    \item Compare the period from your simulation to the analytical expression. Is there a match? 
\end{enumerate}

\section*{Part 2: Damped Oscillator}
\subsection*{Theory}
The damped oscillator equation is given by:
\[ m \frac{d^2x}{dt^2} + b \frac{dx}{dt} + kx = 0 \]
Define \( \beta = \frac{b}{m} \), where \( b \) is the damping coefficient and \( m \) is the mass. The equation can then be rewritten as:
\[ \frac{d^2x}{dt^2} + \beta \frac{dx}{dt} + \frac{k}{m}x = 0 \]

\subsection*{Questions}
\begin{enumerate}
    \item Write the equation in terms of two first-order differential equations. Show your work below:
    \[ \text{Equations: } \]
    \begin{equation*}
    \begin{aligned}
    \text{Our coupled ODEs: } 
    \begin{cases}
    \dfrac{dx}{dt} = v,\\[6pt]
    \dfrac{dv}{dt} = -\beta v - \dfrac{k}{m} x
    \end{cases}
    \end{aligned}
    \end{equation*}
    
    \vspace{0.5cm}
    \item Define mathematically underdamped, overdamped, and critically damped systems. Provide the explanation below:
    \begin{itemize}
    \item \textbf{Underdamped:} \( \beta^2 < \frac{4k}{m} \). Oscillatory motion with exponential decay.
    \item \textbf{Critically damped:} \( \beta^2 = \frac{4k}{m} \). Returns to equilibrium without oscillations in the shortest time.
    \item \textbf{Overdamped:} \( \beta^2 > \frac{4k}{m} \). Returns to equilibrium more slowly than critically damped without oscillations.
\end{itemize}
\end{enumerate}

\subsection*{Tasks}
\begin{enumerate}
    \item Simulate underdamped, overdamped, and critically damped cases by varying $b$.
    \item Plot position $x(t)$ for each case.
    \begin{figure}[H]
    \centering
    \includegraphics[width=0.8\textwidth]{Figure_4.png}
    \caption{Position $x(t)$ of the damped harmonic oscillator for different damping coefficients $b$: Underdamped ($b=0.5$), Critically Damped ($b=2.0$), and Overdamped ($b=10.0$). The system parameters are $m=1.0 \, \text{kg}$ and $k=10.0 \, \text{N/m}$.}
    \label{fig:damped_oscillator}
\end{figure}
    \item Describe how the amplitude decay and frequency change with different $b$ values. Write your observations below:
    \begin{itemize}
        \item As $b$ increases:
        \begin{itemize}
            \item For underdamped cases, the amplitude decays faster, and the frequency decreases slightly.
            \item At critical damping, oscillations disappear entirely.
            \item For overdamped cases, the system takes longer to return to equilibrium.
        \end{itemize}
    \end{itemize}
\end{enumerate}

\section*{Part 3: Driven Damped Oscillator}
\subsection*{Theory}
The driven damped oscillator equation is given by:
\[ m \frac{d^2x}{dt^2} + b \frac{dx}{dt} + kx = F_0 \cos(\omega t) \]
where $F_0$ is the driving force amplitude, and $\omega$ is the driving angular frequency.
Define \( \beta = \frac{b}{m} \), and rewrite the equation as:
\[ \frac{d^2x}{dt^2} + \beta \frac{dx}{dt} + \frac{k}{m}x = \frac{F_0}{m} \cos(\omega t) \]

\subsection*{Questions}
\begin{enumerate}
    \item Write the equation in terms of two first-order differential equations. Show your work below:
    \[
    \text{Equations: }
    \]
    \begin{equation*}
    \begin{aligned}
    \text{Our coupled ODEs: }
    \begin{cases}
    \dfrac{dx}{dt} = v,\\[6pt]
    \dfrac{dv}{dt} = \frac{F_0 \cos(\omega t)}{m} - \beta v - \frac{k}{m} x
    \end{cases}
    \end{aligned}
    \end{equation*}
    \vspace{0.5cm}
    \item How does the driving frequency $\omega$ influence the system's response? Explain the concept of resonance:
    \begin{itemize}
        \item The system has a natural frequency $\omega_0 = \sqrt{\frac{k}{m}}$.
        \item When $\omega$ approaches $\omega_0$, the amplitude of steady-state oscillations increases significantly (resonance).
        \item For $\omega$ far from $\omega_0$, the steady-state amplitude decreases.
    \end{itemize}
    \vspace{0.5cm}
    \item How does the damping coefficient $b$ affect the transient and steady-state behavior of the system?
    \begin{itemize}
        \item \textbf{Transient behavior:} Increasing $b$ causes transient oscillations to decay more quickly.
        \item \textbf{Steady-state behavior:} Higher $b$ reduces the amplitude of the resonance peak and broadens the resonance curve.
    \end{itemize}
    \vspace{0.5cm}
\end{enumerate}

\subsection*{Tasks}
\begin{enumerate}
    \item Simulate the driven damped oscillator for different values of $\omega$ and $b$. Ensure that one case demonstrates resonance.
    \item Plot position $x(t)$ for each case, highlighting both transient and steady-state behavior.
    \begin{figure}[H]
    \centering
    \includegraphics[width=\textwidth]{Figure_5.png} % Replace with actual figure filename
    \caption{Position \(x(t)\) of the driven damped harmonic oscillator for varying driving frequencies (\(\omega\)) with a fixed damping coefficient (\(b=1.0\)). The plot highlights the transient region (shaded in red) and the steady-state region (shaded in blue). Resonance occurs near the natural frequency (\(\omega_{\text{res}} = \sqrt{k/m}\)), leading to the maximum steady-state amplitude.}
    \label{fig:varying_omega}
\end{figure}

\begin{figure}[H]
    \centering
    \includegraphics[width=\textwidth]{Figure_6.png} % Replace with actual figure filename
    \caption{Position \(x(t)\) of the driven damped harmonic oscillator for varying damping coefficients (\(b\)) with a fixed driving frequency (\(\omega = \omega_{\text{res}}\)). The plot highlights the transient region (shaded in red) and the steady-state region (shaded in blue). Stronger damping results in faster decay of transient oscillations and lower steady-state amplitudes, even at resonance.}
    \label{fig:varying_b}
\end{figure}

    \item Describe how the system transitions from transient to steady-state and how damping impacts this transition.

   The driven damped harmonic oscillator undergoes two distinct phases in its motion: transient and steady-state behavior.

\textbf{1. Transient Behavior:}
\begin{itemize}
    \item \textbf{Definition:} The transient phase is the initial phase of motion where the system's behavior is influenced by its initial conditions (e.g., initial position and velocity) and damping.
    \item \textbf{Characteristics:}
    \begin{itemize}
        \item The motion is dominated by the natural frequency of the system, $\omega_{\text{natural}} = \sqrt{\frac{k}{m}}$.
        \item The amplitude of oscillations decays exponentially due to the damping term $-\frac{b}{m} v$ in the governing equation.
    \end{itemize}
    \item \textbf{Duration:} The transient phase typically lasts for a few multiples of the time constant $\tau = \frac{2m}{b}$:
    \begin{itemize}
        \item For smaller damping coefficients ($b$), the transient phase persists longer, resulting in slower decay of oscillations.
        \item For larger damping coefficients ($b$), the transient phase decays faster, leading to quicker transitions.
    \end{itemize}
\end{itemize}

\textbf{2. Steady-State Behavior:}
\begin{itemize}
    \item \textbf{Definition:} After the transient effects decay, the system enters the steady-state phase, where the motion is entirely driven by the external driving force $F \cos(\omega t)$.
    \item \textbf{Characteristics:}
    \begin{itemize}
        \item The oscillations occur at the driving frequency $\omega$, not the natural frequency of the system.
        \item The steady-state amplitude depends on the relationship between $\omega$ (driving frequency) and $\omega_{\text{res}} = \sqrt{\frac{k}{m}}$ (resonance frequency), as well as the damping coefficient $b$.
    \end{itemize}
    \item \textbf{Resonance:} When the driving frequency $\omega$ approaches the resonance frequency $\omega_{\text{res}}$, the system experiences resonance:
    \begin{itemize}
        \item The steady-state amplitude reaches its maximum at resonance, especially for systems with weak damping.
        \item Strong damping suppresses the resonance peak, reducing the amplitude even at $\omega \approx \omega_{\text{res}}$.
    \end{itemize}
\end{itemize}

\textbf{3. Impact of Damping on the Transition:}
Damping plays a crucial role in determining how quickly the system transitions from transient to steady-state and affects the amplitude at steady-state:
\begin{itemize}
    \item \textbf{Weak Damping ($b$ is small):}
    \begin{itemize}
        \item Transient oscillations persist for a longer time due to slower energy dissipation.
        \item The system exhibits a sharp resonance peak, with large steady-state amplitude at $\omega_{\text{res}}$.
    \end{itemize}
    \item \textbf{Moderate Damping:}
    \begin{itemize}
        \item The transient phase decays more quickly, shortening the transition to steady-state.
        \item Resonance is less pronounced, with moderate steady-state amplitude.
    \end{itemize}
    \item \textbf{Strong Damping ($b$ is large):}
    \begin{itemize}
        \item Transient oscillations decay very rapidly, and the system reaches steady-state quickly.
        \item The steady-state amplitude is significantly reduced, even near resonance, as the system dissipates energy more effectively.
    \end{itemize}
\end{itemize}

\begin{figure}[H]
    \centering
    \includegraphics[width=\textwidth]{Figure_7.png} % Replace with actual figure filename
    \caption{Steady-state amplitude of the driven damped harmonic oscillator as a function of driving frequency (\(\omega\)) for a fixed damping coefficient (\(b=1.0\)). The plot demonstrates the resonance peak near the natural frequency (\(\omega_{\text{res}} = \sqrt{k/m}\)), where the amplitude is maximized. The amplitude decreases for driving frequencies far from resonance.}
    \label{fig:amplitude_vs_omega}
\end{figure}

\begin{figure}[H]
    \centering
    \includegraphics[width=\textwidth]{Figure_8.png} % Replace with actual figure filename
    \caption{Steady-state amplitude of the driven damped harmonic oscillator as a function of damping coefficient (\(b\)) with a fixed driving frequency near resonance (\(\omega = \omega_{\text{res}}\)). The plot shows that stronger damping suppresses the steady-state amplitude, reducing the resonance effect.}
    \label{fig:amplitude_vs_b}
\end{figure}
    
    \vspace{3cm}
\end{enumerate}

\end{document}
